%\documentclass[11pt]{article}
\documentclass[11pt]{scrartcl}
%\usepackage{a4paper}
\usepackage{fullpage}
\usepackage{longtable}
\usepackage{bytefield}
\usepackage{color}
\usepackage{pslatex}

\title{PHCC Programmers Guide - Talking to DOA devices}
\author{Manuel Bessler}
\date{24 April 2005}

% Set up hyperlink colors
\definecolor{darkred}{rgb}{0.5,0,0}
\definecolor{darkgreen}{rgb}{0,0.3,0}
\definecolor{darkblue}{rgb}{0,0,0.5}
\definecolor{darkbrown}{rgb}{0.28,0.07,0.07}

\setlength{\parindent}{0pt} % don't indent paragraphs 

\begin{document}
\sloppy
\maketitle

\begin{abstract}
This document is targetted at both hardware and software developers.
It describes the data packets and board-specific configuration variables
for the PHCC DOA (Digital Out Type A) daughterboards.
\end{abstract}

%%%%%%%%%%%%%%%%%%%%%%%%%%%%%%%%%%%%%%%%%%%%%%%%%%%%%%%%%%%%%%%%%%%%%%%%

%%\setcounter{section}{1}

\section{Protocol Overview}

Communication with PHCC always involves the motherboard as the central "hub"
of the PHCC system. On the hardware level, data is transferred back and forth
between the host (PC) and PHCC via:
\begin{itemize}
\item serial port
\item USB port
\item (maybe in the future) ethernet
\end{itemize}

Data transfer to or from PHCC is broken down into \textbf{packets}.
A \textbf{packet} consists of one or more \textbf{bytes}.
In this document we shall only regard packets \textsl{from the host to PHCC}
as DOA transfers are one way only.

\begin{figure}[htbp] \begin{center}
\setlength{\byteheight}{6ex}
\setlength{\bitwidth}{1cm}
\begin{bytefield}{8}
\wordbox{1}{command byte} \\
\wordbox{1}{optional payload data byte 0} \\
\wordbox[]{1}{$\vdots$ \\[1ex]} \\
\wordbox{1}{optional payload data byte N}
\end{bytefield}
\caption{Packet Layout}
\label{packet layout}
\end{center} \end{figure}

As you can see from Figure \ref{packet layout}, only the \textbf{command byte} 
is required in any case. The number of \textbf{payload data bytes} that are 
expected depends on the command byte.

Figure \ref{command bytes} lists a few common \textbf{command bytes}.

\begin{figure}[htbp] \begin{center}
\begin{tabular}{l|l}
Command Byte & Meaning \\
\hline
0x00 & IDLE \\
0x01 & RESET \\
0x02 & START TALKING \\
0x03 & STOP TALKING \\
0x04 & dump keymatrix map \\
0x05 & dump analog map \\
0x06 & I2C send \\
0x07 & DOA send \\
0x08 & DOB send \\
\end{tabular}
\caption{common command bytes}
\label{command bytes}
\end{center} \end{figure}

For this document, we will only concern ourselves with 
\textbf{DOA send} (command byte 0x07\footnote{The 0xnn notation is used here 
as in the C programming language. It refers to numbers in the hexadecimal system.}).


\section{The DOA Command Packet}

The \textbf{DOA Command Packet} consists of the \textbf{DOA send command byte} 
and three bytes as payload:


When PHCC receives a \textbf{DOA send command byte}, it expects three bytes to
follow the \textbf{command bytes}. See figure \ref{doa send}.

\begin{figure}[h] \begin{center}
\setlength{\byteheight}{6ex}
\setlength{\bitwidth}{1cm}
\begin{bytefield}{8}
\wordbox{1}{command byte} \\
\wordbox{1}{device address (devaddr)} \\
\wordbox{1}{subdevice address (subaddr)} \\
\wordbox{1}{DOA data byte}
\end{bytefield}
\caption{DOA Command Packet Layout}
\label{doa send}
\end{center} \end{figure}


\subsection{Device Address}

The device address (or \textbf{devaddr}) is a 8bit value to address 
individual DOA boards. In most cases, each DOA board will have a unique
\textbf{devaddr}. When the firmware for a particular DOA daughterboard 
is compiled or assembled (using a C compiler or PIC assembler) this address
is set\footnote{usually via a \texttt{\#define}}. The 8bit width means that 
256 DOA daughterboards could be theoretically addressed.

In certain cases, a single DOA daughterboard might require more than 
one \textbf{devaddr}, for example when the number of subaddresses is not 
sufficient or in case of other special requirements.

\subsection{Subdevice Address}

The subdevice address or subaddress (or \textbf{subaddr}) is a 6bit value
to address certain functions on a particular DOA daughterboard. Note that while
this is a 6bit value, the smallest unit that can be transferred via the serial port 
is a byte (8bit). This means that when the host sends a \textbf{subaddr} to PHCC,
it will be the size of a byte, with the two most significant bits unset. 
PHCC ignores the value of those two bits.

The size of the \textbf{subaddr} might change in the future, depending on needs.

The interpretation of a \textbf{subaddr} depends on the type of DOA daughterboard
and possibly also firmware version. Further below (see subaddress maps, starting 
figure \ref{subaddr map 40do})
is a list of subaddresses and their meaning for certain board firmware versions.


\subsection{DOA Data Byte}

The \textbf{data byte} is a 8bit (well of course, its a byte) value whose meaning is 
determined by the firmware of the receiving daughterboard in combination with 
the \textbf{subaddr}

Detailed information is presented in the subadress maps section.


\section{Subaddress Maps}

The following figures show maps for individual DOA daughterboards.
The subaddr is given in hexadecimal format along with the meaning
of each particular subaddr.


\subsection{Subaddress Map for Daughterboard PHCC\_DOA\_40DO}

The valid subaddresses for the 40DO board are shown in figure \ref{subaddr map 40do}.
The 40 outputs are bit-mapped onto the subaddresses. Each byte contains 8bits, 
so each subaddr corresponds to 8 outputs.

\begin{figure}[ht] \begin{center}
\setlength{\byteheight}{4ex}
\setlength{\bitwidth}{0.25cm}
\begin{bytefield}{32}
\bitbox[]{8}{Subaddr} & \bitbox[]{24}{Description} \\
\bitbox[]{8}{\texttt{0x00}} & \bitbox{24}{outputs 1-8} \\
\bitbox[]{8}{\texttt{0x01}} & \bitbox{24}{outputs 9-16} \\
\bitbox[]{8}{\texttt{0x02}} & \bitbox{24}{outputs 17-24} \\
\bitbox[]{8}{\texttt{0x03}} & \bitbox{24}{outputs 25-32} \\
\bitbox[]{8}{\texttt{0x04}} & \bitbox{24}{outputs 33-40} \\
\end{bytefield}
\caption{Subaddr Map for daughterboard DOA\_40DO}
\label{subaddr map 40do}
\end{center} \end{figure}



\subsection{Subaddress Map for Daughterboard PHCC\_DOA\_7seg}

The valid subaddresses for the LED/7segment display driver board DOA\_7seg
are shown in figure \ref{subaddr map 7seg}.
Up to 32 displays can be controlled with one board, so 32 subaddresses are
needed for direct control, one for each display. Most 7-segment displays
actually have 8 LEDs embedded (7 segments and the decimal point) which suits
us well as each byte also consists of 8 bits. This board allows direct control
of all segments (as well as the decimal point), so no BCD or other conversion 
is done on the incoming data. It is the host's responsibility to ,,encode'' the
data eg. to display numbers 0-9 via the 7 segments. This also give the user 
freedom in wiring of the segmented displays. Instead of (or in addition to)
the 7-segment displays, individual LEDs can also be used. In this case, it is necessary
to connect the cathodes of every block of eight LEDs together to form a common cathode
(if the board is used in common cathode configuration, otherwise connect the anodes together).

The board has two sets of 16 displays grouped together as a ,,channel''. 
The purpose of these channels is to group the commons together. This means that the common
return of display 1 (on channel A) is connected to the first display of channel B, 
namely display 17.


\begin{figure}[ht] \begin{center}
\setlength{\byteheight}{3ex}
\setlength{\bitwidth}{0.25cm}
\begin{bytefield}{24}
\bitbox[]{8}{Subaddr} & \bitbox[]{24}{Description} \\
\wordgroupr{Displays 1-16 \\ on Channel A}
\bitbox[]{8}{\texttt{0x00}} & \bitbox{24}{display 1} \\
\bitbox[]{8}{\texttt{0x01}} & \bitbox{24}{display 2} \\
\bitbox[]{8}{\texttt{0x02}} & \bitbox{24}{display 3} \\
\bitbox[]{8}{\texttt{0x03}} & \bitbox{24}{display 4} \\
\bitbox[]{8}{\texttt{0x04}} & \bitbox{24}{display 5} \\
\bitbox[]{8}{\texttt{0x05}} & \bitbox{24}{display 6} \\
\bitbox[]{8}{\texttt{0x06}} & \bitbox{24}{display 7} \\
\bitbox[]{8}{\texttt{0x07}} & \bitbox{24}{display 8} \\
\bitbox[]{8}{\texttt{0x08}} & \bitbox{24}{display 9} \\
\bitbox[]{8}{\texttt{0x09}} & \bitbox{24}{display 10} \\
\bitbox[]{8}{\texttt{0x0A}} & \bitbox{24}{display 11} \\
\bitbox[]{8}{\texttt{0x0B}} & \bitbox{24}{display 12} \\
\bitbox[]{8}{\texttt{0x0C}} & \bitbox{24}{display 13} \\
\bitbox[]{8}{\texttt{0x0D}} & \bitbox{24}{display 14} \\
\bitbox[]{8}{\texttt{0x0E}} & \bitbox{24}{display 15} \\
\bitbox[]{8}{\texttt{0x0F}} & \bitbox{24}{display 16} 
\endwordgroupr \\
\wordgroupr{Displays 17-32 \\ on Channel B}
\bitbox[]{8}{\texttt{0x10}} & \bitbox{24}{display 17} \\
\bitbox[]{8}{\texttt{0x11}} & \bitbox{24}{display 18} \\
\bitbox[]{8}{\texttt{0x12}} & \bitbox{24}{display 19} \\
\bitbox[]{8}{\texttt{0x13}} & \bitbox{24}{display 20} \\
\bitbox[]{8}{\texttt{0x14}} & \bitbox{24}{display 21} \\
\bitbox[]{8}{\texttt{0x15}} & \bitbox{24}{display 22} \\
\bitbox[]{8}{\texttt{0x16}} & \bitbox{24}{display 23} \\
\bitbox[]{8}{\texttt{0x17}} & \bitbox{24}{display 24} \\
\bitbox[]{8}{\texttt{0x18}} & \bitbox{24}{display 25} \\
\bitbox[]{8}{\texttt{0x19}} & \bitbox{24}{display 26} \\
\bitbox[]{8}{\texttt{0x1A}} & \bitbox{24}{display 27} \\
\bitbox[]{8}{\texttt{0x1B}} & \bitbox{24}{display 28} \\
\bitbox[]{8}{\texttt{0x1C}} & \bitbox{24}{display 29} \\
\bitbox[]{8}{\texttt{0x1D}} & \bitbox{24}{display 30} \\
\bitbox[]{8}{\texttt{0x1E}} & \bitbox{24}{display 31} \\
\bitbox[]{8}{\texttt{0x1F}} & \bitbox{24}{display 32} 
\endwordgroupr \\
\end{bytefield}
\caption{Subaddr Map for daughterboard DOA\_7seg}
\label{subaddr map 7seg}
\end{center} \end{figure}



\subsection{Subaddress Map for Daughterboard PHCC\_DOA\_8servo}

The valid subaddresses for the servo driver board DOA\_8servo
are shown in figure \ref{subaddr map servo}.
To control these 8 RC-type servos, there are subadresses for 
position and calibration.
The position is a 8bit value, so 256 different positions are possible
for each servo.
Each servo can be calibrated for movement range and for offset 
on one side of the movement range.
The offset calibration value is 16bits, so two subadresses are
needed for each servo.
The movement range calibration value (also called ,,gain'') is a 8bit value.
The best strategy for calibration is to set the offset value first, and then 
the gain.


\begin{figure}[ht] \begin{center}
\setlength{\byteheight}{3ex}
\setlength{\bitwidth}{0.4cm}
\begin{bytefield}{24}
\bitbox[]{8}{Subaddr} & \bitbox[]{24}{Description} \\
\wordgroupr{Servo Positions}
\bitbox[]{8}{\texttt{0x00}} & \bitbox{24}{servo 1} \\
\bitbox[]{8}{\texttt{0x01}} & \bitbox{24}{servo 2} \\
\bitbox[]{8}{\texttt{0x02}} & \bitbox{24}{servo 3} \\
\bitbox[]{8}{\texttt{0x03}} & \bitbox{24}{servo 4} \\
\bitbox[]{8}{\texttt{0x04}} & \bitbox{24}{servo 5} \\
\bitbox[]{8}{\texttt{0x05}} & \bitbox{24}{servo 6} \\
\bitbox[]{8}{\texttt{0x06}} & \bitbox{24}{servo 7} \\
\bitbox[]{8}{\texttt{0x07}} & \bitbox{24}{servo 8} 
\endwordgroupr \\
\wordgroupr{Offset Calibration \\ value \\ (low byte)}
\bitbox[]{8}{\texttt{0x08}} & \bitbox{24}{low byte of calibration offset for servo 1} \\
\bitbox[]{8}{\texttt{0x09}} & \bitbox{24}{low byte of calibration offset for servo 2} \\
\bitbox[]{8}{\texttt{0x0A}} & \bitbox{24}{low byte of calibration offset for servo 3} \\
\bitbox[]{8}{\texttt{0x0B}} & \bitbox{24}{low byte of calibration offset for servo 4} \\
\bitbox[]{8}{\texttt{0x0C}} & \bitbox{24}{low byte of calibration offset for servo 5} \\
\bitbox[]{8}{\texttt{0x0D}} & \bitbox{24}{low byte of calibration offset for servo 6} \\
\bitbox[]{8}{\texttt{0x0E}} & \bitbox{24}{low byte of calibration offset for servo 7} \\
\bitbox[]{8}{\texttt{0x0F}} & \bitbox{24}{low byte of calibration offset for servo 8} 
\endwordgroupr \\
\wordgroupr{Offset Calibration \\ value \\ (high byte)}
\bitbox[]{8}{\texttt{0x10}} & \bitbox{24}{high byte of calibration offset for servo 1} \\
\bitbox[]{8}{\texttt{0x11}} & \bitbox{24}{high byte of calibration offset for servo 2} \\
\bitbox[]{8}{\texttt{0x12}} & \bitbox{24}{high byte of calibration offset for servo 3} \\
\bitbox[]{8}{\texttt{0x13}} & \bitbox{24}{high byte of calibration offset for servo 4} \\
\bitbox[]{8}{\texttt{0x14}} & \bitbox{24}{high byte of calibration offset for servo 5} \\
\bitbox[]{8}{\texttt{0x15}} & \bitbox{24}{high byte of calibration offset for servo 6} \\
\bitbox[]{8}{\texttt{0x16}} & \bitbox{24}{high byte of calibration offset for servo 7} \\
\bitbox[]{8}{\texttt{0x17}} & \bitbox{24}{high byte of calibration offset for servo 8} 
\endwordgroupr \\
\wordgroupr{Gain Calibration \\ value}
\bitbox[]{8}{\texttt{0x18}} & \bitbox{24}{gain calibration value for servo 1} \\
\bitbox[]{8}{\texttt{0x19}} & \bitbox{24}{gain calibration value for servo 2} \\
\bitbox[]{8}{\texttt{0x1A}} & \bitbox{24}{gain calibration value for servo 3} \\
\bitbox[]{8}{\texttt{0x1B}} & \bitbox{24}{gain calibration value for servo 4} \\
\bitbox[]{8}{\texttt{0x1C}} & \bitbox{24}{gain calibration value for servo 5} \\
\bitbox[]{8}{\texttt{0x1D}} & \bitbox{24}{gain calibration value for servo 6} \\
\bitbox[]{8}{\texttt{0x1E}} & \bitbox{24}{gain calibration value for servo 7} \\
\bitbox[]{8}{\texttt{0x1F}} & \bitbox{24}{gain calibration value for servo 8} 
\endwordgroupr \\
%\wordgroupr{(not implemented yet)}
\bitbox[]{8}{\texttt{0x20}} & \bitbox[lrt]{24}{store calibration values to EEPROM} \\
\bitbox[]{8}{} & \bitbox[lrb]{24}{(not implemented yet)}
%\endwordgroupr \\
\end{bytefield}
\caption{Subaddr Map for daughterboard DOA\_8servo}
\label{subaddr map servo}
\end{center} \end{figure}


\subsection{Subaddress Map for Daughterboard PHCC\_DOA\_AnOut1}

The valid subaddresses for the servo driver board DOA\_AnOut1
are shown in figure \ref{subaddr map anout1}.
With the standard firmware, each channel has a resolution of 8bits.
For non-standard firmware versions (eg. with higher resolution but
less channels) this subaddr map is not correct.
Note the \textbf{gain calibration value} at subaddr 0x10, which is 
currently preset to 128 (decimal). Normally the gain shouln't 
need adjustment and it might completely disappear completely in later
firmware versions.

\begin{figure}[ht] \begin{center}
\setlength{\byteheight}{4ex}
\setlength{\bitwidth}{0.25cm}
\begin{bytefield}{32}
\bitbox[]{8}{Subaddr} & \bitbox[]{24}{Description} \\
\bitbox[]{8}{\texttt{0x00}} & \bitbox{24}{channel 1} \\
\bitbox[]{8}{\texttt{0x01}} & \bitbox{24}{channel 2} \\
\bitbox[]{8}{\texttt{0x02}} & \bitbox{24}{channel 3} \\
\bitbox[]{8}{\texttt{0x03}} & \bitbox{24}{channel 4} \\
\bitbox[]{8}{\texttt{0x04}} & \bitbox{24}{channel 5} \\
\bitbox[]{8}{\texttt{0x05}} & \bitbox{24}{channel 6} \\
\bitbox[]{8}{\texttt{0x06}} & \bitbox{24}{channel 7} \\
\bitbox[]{8}{\texttt{0x07}} & \bitbox{24}{channel 8} \\
\bitbox[]{8}{\texttt{0x08}} & \bitbox{24}{channel 9} \\
\bitbox[]{8}{\texttt{0x09}} & \bitbox{24}{channel 10} \\
\bitbox[]{8}{\texttt{0x0a}} & \bitbox{24}{channel 11} \\
\bitbox[]{8}{\texttt{0x0b}} & \bitbox{24}{channel 12} \\
\bitbox[]{8}{\texttt{0x0c}} & \bitbox{24}{channel 13} \\
\bitbox[]{8}{\texttt{0x0d}} & \bitbox{24}{channel 14} \\
\bitbox[]{8}{\texttt{0x0e}} & \bitbox{24}{channel 15} \\
\bitbox[]{8}{\texttt{0x0f}} & \bitbox{24}{channel 16} \\
\bitbox[]{8}{\texttt{0x10}} & \bitbox[lrt]{24}{gain calibration value} \\
\bitbox[]{8}{} & \bitbox[lrb]{24}{for all channels}
\end{bytefield}
\caption{Subaddr Map for daughterboard DOA\_AnOut1}
\label{subaddr map anout1}
\end{center} \end{figure}


\subsection{Subaddress Map for Daughterboard PHCC\_DOA\_char\_lcd}

TBD.
Standard (generic) firmware:
Here, the subaddr is divided into several fields, namely
* display number (0-7): 3bit, contained in the lower 3 bits (bits 0-2)
* display data/control data 0/1: 1bit, contained in bit 333 
  display data=0, control data=1

The lcd firmware buffers the incoming data/commands in a circular buffer.
This buffer can hold up to 128 data/command sequences


\subsection{Subaddress Map for Daughterboard PHCC\_DOA\_stepper\_293}

TBD.


\end{document}
